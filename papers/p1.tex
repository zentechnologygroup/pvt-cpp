
\documentclass[9pt,conference]{IEEEtran}

\usepackage[utf8]{inputenc}
\usepackage[T1]{fontenc} % optional
\usepackage{amsmath}
\usepackage[cmintegrals]{newtxmath}
\usepackage{bm} % optiona

\title{ZEN PVT Toolbox}

\author{Leandro Rabindranath León \\
       Virginia  Buccellato \\
       Ixhel Mejias \\
       Alberto Valderrama\\
       Neylith Quintero\\
       Fernando Montilla\\
       Eduardo Valderrama\\
       Felix Buccellato}

\begin{document}

\maketitle

 \section{Introducción}

 Una adecuada y precisa caracterización de las diferentes propiedades
 físicas de un fluido (relación gas/petróleo, factor volumétrico,
 viscosidad, etc.) es cŕitica para la optimización de la
 producción. Simplemente, no es posible correctamente diseñar un sistema
 de explotación, ni operarlo, si no se conoce con precisión el
 comportamiento del fluido ante variaciones de temperatura y presión.

 Ahora bien, a la necesidad anterior se antepone un gran problema: 

  \subsection{El problema}

  \subsection{Las metas de este trabajo}

  \subsection{El enfoque}

 \section{Estado del arte}

 \section{El enfoque PVT toolbox de ZEN}

  \subsection{Arquitectura}

  \subsection{Separación del back end y front end}

  \subsection{Implementación}

  \subsection{Desempeño}

 \section{Conclusión}





\end{document}